%%%%%%%%%%%%%%%%%%%%%%%%%%%%%%%%%%%%%%%%%%%%%%%%%%%%%%%%%%%%%%%%%%%%%%%%%%%
%
% Plantilla para un art�culo en LaTeX en espa�ol.
%
%%%%%%%%%%%%%%%%%%%%%%%%%%%%%%%%%%%%%%%%%%%%%%%%%%%%%%%%%%%%%%%%%%%%%%%%%%%

\documentclass{article}

% Esto es para poder escribir acentos directamente:
\usepackage[utf8]{inputenc}
\usepackage[spanish]{babel}
% Paquetes de la AMS:
\usepackage{amsmath, amsthm, amsfonts, amssymb}

% Paquete para enumerar
\usepackage{enumerate}

% Teoremas
%--------------------------------------------------------------------------
\newtheorem*{teorema*}{Teorema}
\newtheorem{teorema}{Teorema}

% Atajos.
%------------------------------------------------------------------------
\newcommand{\R}{\mathbb{R}}
\newcommand{\N}{\mathbb{N}}
%-------------------------------------------------------------------------
\title{Ejercicio 16}
\author{Blanca Cano Camarero}

\begin{document}
\maketitle

\begin{teorema*}
    Sea $A$ un dominio de ideales principales. 
    Se tiene que
    \begin{enumerate}[i]
        \item ${_A}A$ es de longitud finita si y sólo si A es un cuerpo.
        \item Si $I$ es un ideal no nulo de $A$, entonces el $A-$módulo $A/I$ es de longitud finita. 
    \end{enumerate}
    Debe de preguntarse uno también si se sabe algo sobre la longitud y los factores de composición de $A/I$ de un generador del ideal $I$?.
\end{teorema*}   

\begin{proof}
    hola esto es una demostración.
\end{proof}





\end{document}