%%%%%%%%%%%%%%%%%%%%%%%%%%%%%%%%%%%%%%%%%%%%%%%%%%%%%%%%%%%%%%%%%%%%%%%%%%%
%
% Plantilla para un art�culo en LaTeX en espa�ol.
%
%%%%%%%%%%%%%%%%%%%%%%%%%%%%%%%%%%%%%%%%%%%%%%%%%%%%%%%%%%%%%%%%%%%%%%%%%%%

\documentclass{article}

% Esto es para poder escribir acentos directamente:
\usepackage[utf8]{inputenc}
\usepackage[spanish]{babel}
% Paquetes de la AMS:
\usepackage{amsmath, amsthm, amsfonts, amssymb}

% Paquete para enumerar
\usepackage{enumerate}

% Teoremas
%--------------------------------------------------------------------------
\newtheorem*{teorema*}{Teorema}
\newtheorem{teorema}{Teorema}

% Atajos.
%------------------------------------------------------------------------
\newcommand{\R}{\mathbb{R}}
\newcommand{\N}{\mathbb{N}}
%-------------------------------------------------------------------------
\title{Ejercicio 16}
\author{Blanca Cano Camarero}

\begin{document}
\maketitle

\begin{teorema*}
    Sea $A$ un dominio de ideales principales. 
    Se tiene que
    \begin{enumerate}[i]
        \item ${_A}A$ es de longitud finita si y sólo si A es un cuerpo.
        \item Si $I$ es un ideal no nulo de $A$, entonces el $A-$módulo $A/I$ es de longitud finita. 
    \end{enumerate}
    Añadir comentario si se sabe algo sobre la longitud y los factores de composición de $A/I$ de un generador del ideal $I$?.
\end{teorema*}   

\begin{proof}
(I) Condición suficiente, Por ser $A$ un dominio de ideales principales sabemos que es conmutativo, ahora bastará probar que es unitario. 

Para cualquier $a \in A$ distinto del cero se tiene que 
$a A = A$ luego existirá $a' \in A$ tal que 
$a a'=1$ y puesto que es conmutativo se tiene que también 
$a' a = a a'=1$, es decir, que todo elemento distinto de cero tiene inverso. 

(I) Condición necesaria, por ser un dominio de ideales principales 
$A = <a>$ con $a \in A$, además $0 \in A$ por ser un cuerpo, y entonces existirá $n \in \N \setminus \{0\}$ tal que $a^n = 0.$

Así pues podemos considerar la siguiente serie de composición 
\begin{equation*}
    A = <a> \supset 
    <a^2> 
    \supset 
    \ldots 
    <a^{n-1}>
    \supset
    <a^n> = \{0\}.
\end{equation*}

(II) Si $I$ es impropio, por ser no vacío entonces necesariamente 
$I=A$ y está claro que el cociente $A/I$ es simple y tendrá longitud 1. 

En caso de ser $I$ un ideal propio, observemos que 
$A$ es un ideal de $A$ y por estar en un DIP $A=<a>$, por lo que 
necesariamente $I= <a^e>$ con $e > 1$, entonces consideramos la cadena

\begin{equation*}
    \frac{<a>}{<a^e>} \supset  \frac{<2a>}{<a^e>} 
    \supset \ldots \supset
    \frac{<a^e -a>}{<a^e>}
    \supset
    \frac{<a^e>}{<a^e>} = \{0\}
\end{equation*}
\end{proof}





\end{document}