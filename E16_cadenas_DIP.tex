%%%%%%%%%%%%%%%%%%%%%%%%%%%%%%%%%%%%%%%%%%%%%%%%%%%%%%%%%%%%%%%%%%%%%%%%%%%
%
% Plantilla para un art�culo en LaTeX en espa�ol.
%
%%%%%%%%%%%%%%%%%%%%%%%%%%%%%%%%%%%%%%%%%%%%%%%%%%%%%%%%%%%%%%%%%%%%%%%%%%%

\documentclass{article}

% Esto es para poder escribir acentos directamente:
\usepackage[utf8]{inputenc}
\usepackage[spanish]{babel}
% Paquetes de la AMS:
\usepackage{amsmath, amsthm, amsfonts, amssymb}

% Paquete para enumerar
\usepackage{enumerate}

% Teoremas
%--------------------------------------------------------------------------
\newtheorem*{teorema*}{Teorema}
\newtheorem{teorema}{Teorema}

% Atajos.
%------------------------------------------------------------------------
\newcommand{\R}{\mathbb{R}}
\newcommand{\N}{\mathbb{N}}
%-------------------------------------------------------------------------
\title{Ejercicio 16}
\author{Blanca Cano Camarero}

\begin{document}
\maketitle

\begin{teorema*}
    Sea $A$ un dominio de ideales principales. 
    Se tiene que
    \begin{enumerate}[i]
        \item ${_A}A$ es de longitud finita si y sólo si A es un cuerpo.
        \item Si $I$ es un ideal no nulo de $A$, entonces el $A-$módulo $A/I$ es de longitud finita. 
    \end{enumerate}
    Añadir comentario si se sabe algo sobre la longitud y los factores de composición de $A/I$ de un generador del ideal $I$?.
\end{teorema*}   

\begin{proof}
(I) Condición suficiente. Por ser $A$ un dominio de ideales principales sabemos que es conmutativo, ahora bastará probar que es unitario. 

Para cualquier $a \in A$ distinto del cero se tiene que 
$a A = A$ luego existirá $a' \in A$ tal que 
$a a'=1$ y puesto que es conmutativo se tiene que también 
$a' a = a a'=1$, es decir, que todo elemento distinto de cero tiene inverso. 

(I) Condición necesaria, por ser un cuerpo los únicos ideales que tiene son los impropios. Por ser conmutativo, los ideales serán particularmente ideales a izquierda, es decir los submódulos de $A$, luego la única serie de composición admisible es la simple. 
\begin{equation*}
    A = <a> \supset \{0\}.
\end{equation*}

(II) Si $I$ es impropio, por ser no vacío entonces necesariamente 
$I=A$ y está claro que el cociente $A/I$ es simple y tendrá longitud 1. 

En caso de ser $I$ un ideal propio, estará generado por $<\mu>$ que a su vez admitirá una factorización única  $\mu = \prod p_i^{s_i}$, a su vez $A=<\lambda> = < \prod p_i ^{r_i}>$.  
Definimos de manera inducctiva a continuación la aplicación 
\begin{equation*}
    \sigma: \N \times A \longrightarrow \N
\end{equation*}
tal que $\sigma(0,p_i) = r_i$
y donde $\sigma(n+1, \cdot)$ viene determinado de la siguiente manera: 
\begin{itemize}
    \item Si existe un  $\sigma(n, p_i) < s_i$ entonces $\sigma(n+1, p_i) = \sigma(n, p_i)+1$ y para el resto de factores $\sigma(n+1, p_j) = \sigma(n, p_j)$.
    \item  En caso de no encontrar entonces $\sigma(n+1, p_i) = \sigma(n, p_i)$ para cualquier factor.
\end{itemize}

Bajo esta definición está claro que la definición no es única, pero eso no es un impedimento para que esté bien definida.

A partir de cierto natural $m$, cualquier $w>m$ $\sigma(w, p_i) = s_i$ sea cual sea el factor $p_i$ tomado. 
Construímos entonces la siguiente serie de composición
\begin{equation*} 
    \frac{A}{I} = 
    \frac{< \prod p_i ^{\sigma(0, p_i)}>}{I} 
    \supset  
    \frac{< \prod p_i ^{\sigma(1, p_i)}>}{I} 
    \supset 
    \ldots 
    \supset 
    \frac{< \prod p_i ^{\sigma(m, p_i)}>}{I}  
    = \frac{I}{I} 
    = 
    \{0\}
\end{equation*}


A la vista de esta construcción está claro que la longitud será el mínimo $m$ tal que 
$\sigma(m, \cdot)= \sigma(m+1, \cdot)$
es decir el número de factores $p_i^{e_i}$ que $p_i^{e_i} \lambda | \mu$
y la forma de construirlo viene determinada por $\sigma$. 
\end{proof}





\end{document}