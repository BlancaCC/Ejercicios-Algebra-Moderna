%%%%%%%%%%%%%%%%%%%%%%%%%%%%%%%%%%%%%%%%%%%%%%%%%%%%%%%%%%%%%%%%%%%%%%%%%%%
%
% Plantilla para un art�culo en LaTeX en espa�ol.
%
%%%%%%%%%%%%%%%%%%%%%%%%%%%%%%%%%%%%%%%%%%%%%%%%%%%%%%%%%%%%%%%%%%%%%%%%%%%

\documentclass{article}

% Esto es para poder escribir acentos directamente:
\usepackage[utf8]{inputenc}
\usepackage[spanish]{babel}


% Paquetes de la AMS:
\usepackage{amsmath, amsthm, amsfonts, amssymb}

% Teoremas
%--------------------------------------------------------------------------
\newtheorem*{teorema*}{Teorema}
\newtheorem{teorema}{Teorema}

% Atajos.
%------------------------------------------------------------------------

%-------------------------------------------------------------------------
\title{Ejercicio 6}
\author{Blanca Cano Camarero}

\begin{document}
\maketitle

\begin{teorema*}
    Sea $A$ un anillo, $A$ es un anillo de división sii tiene exactamente dos ideales a izquierda. 
\end{teorema*}   
\begin{proof}
 

\subsubsection*{Condición suficiente:}

Por definición $N$ es un submódulo del módulo regular de $A$ si 
para cualquier $m\in N$ se tiene que $am \in N$ sea cual sea $a \in A$. 

Bajo esta definición está claro que $\{0\}$ es un submódulo del módulo regular. Supongamos otro submódulo no trivial $N$
con $a \in N$ diferente del $0$, entonces puesto que es un anillo unitario  existirá $a^{-1} \in A$, de esta manera se tiene que 
$a^{-1} a \in N.$
Aplicando de nuevo que es un submódulo se tiene que 
$a 1 \in N$ para todo $a \in A$, esto es $N=A$. 

Es decir acabamos de probar que cualquier submódulo no trivial es necesariamente el total, por lo que concluímos que tiene dos submódulo. 

\subsubsection*{Condición necesaria:} 

Está claro que $\{0\}$ y $A$ son submódulos del módulo regular $A$, 
puesto que tiene solo dos ideales a la izquierda deben de ser ellos.  

Tomamos $a \in A$ distinto de cero y queremos ver que tiene inverso. Por se $A$ 
un ideal a la izquierda 
$Aa = A$ por lo que existirá un elemento $a' \in A$ tal que 
$a' a = 1$. 
Por otro lado repetimos el mismo argumento $Aa' = A$ por lo que 
existe $a'' \in A$ tal que $a''a' = 1$,
De aquí se deduce que 
\begin{align*}
    a'a = 1 
    \Leftrightarrow  
    a''(a'a) = a''1
    \Leftrightarrow 
    a = a''.
\end{align*}
Por lo que acabamos de ver que 
\begin{align*}
    a'a = 1 = aa',
\end{align*}
es decir que $a' = a^{-1}$ como queríamos probar. 
\end{proof}
\end{document}