\documentclass{article}

% Esto es para poder escribir acentos directamente:
\usepackage[utf8]{inputenc}
\usepackage[spanish]{babel}


% Paquetes de la AMS:
\usepackage{amsmath, amsthm, amsfonts, amssymb}

% Teoremas
%--------------------------------------------------------------------------
\newtheorem*{teorema*}{Teorema}
\newtheorem{teorema}{Teorema}

% Atajos.
%------------------------------------------------------------------------
\newcommand{\R}{\mathbb{R}}
\newcommand{\N}{\mathbb{N}}
%-------------------------------------------------------------------------
\title{Ejercicio 4 álgebra moderna}
\author{Blanca Cano Camarero}

\begin{document}
\maketitle

\begin{teorema*}
    Sea $T:V \longrightarrow V$ una transformación lineal en un $K-$espacio vectorial de dimensión finita $V$.
    Se tiene que si $V \neq \{0\}$ y $\{0\}$ y $V$ son los únicos 
    subespacios $T$-invariantes de $V$, entonces el polinomio mínimo de $T$ es irreducible en $K[X]$. El recíproco no es cierto. 
\end{teorema*}   

\begin{proof}

    Se tiene que $V$ es un $K[X]$-módulo definido por para cada
    $p \in K[X]$
    \begin{equation*}
        p \cdot v = p (T(v)).
    \end{equation*}
   Sea $\mu$ el polinomio mínimo de $T$, razonaremos por contradicción. 
   Si $T$ admite una descomposición en factores de la forma
   $\mu = \prod^2_i p_i$ . 
   Para cada $i \in \{1, 2\}$ llamamos  $q_i = \frac{\mu}{p_i}$ y vemos que define un subespacio vectorial sobre $K$
   \begin{equation*}
       V_i = \{ v \in V : q_i \cdot v = 0\}
   \end{equation*}
   ya que
   \begin{itemize}
       \item Para cualquiera $u,v \in V_i$ se cumple que 
       $q_i \cdot (u + v) = q_i \cdot u +q_i \cdot v = 0$,
        por lo que $u+v  \in V_i$. 
        \item Para cualquier $u \in V_i$ y $k \in K$ podemos ver $k$ como un elemento del  $K[x]-$módulo definido por 
        $x \mapsto kT(x)$, de esta manera: 
        $q_i \cdot (k \cdot u) = (q_i \cdot k )\cdot u = k \cdot q_i \cdot u =  k \cdot 0 = 0.$
   \end{itemize}
    Además son $T$-invariantes puesto que tomando  
    $v \in V_i$ cualquiera 
   \begin{equation*}
       q_i \cdot T(v) 
       = q_i \cdot x \cdot  v 
       = x \cdot q_i \cdot v 
       = x \cdot 0
       = T(0)
       = 0. 
   \end{equation*}

Por hipótesis los únicos subespacios $T$-invariantes son los triviales:

\begin{enumerate}
    \item Si $V_i = V$ entonces $Ann_{K[K]}(V)= <q_i>$, lo cual contradice que $\mu$ sea polinomio mínimo, por lo que 
    $V_i=\{0\}$ para cualquier $i \in \{1, 2\}$. 
    \item Si $V_i=\{0\}$ consideramos $j \neq i$ y tenemos que 
    $0 = \mu \cdot v =  q_i \cdot q_j \cdot v = 0$ sii $q_j \cdot v = 0$ por lo que $Ann_{K[K]}(V)= <q_j>$ lo cual vuelve a contradecir que $\mu$ sea mínimo.
\end{enumerate}

Concluimos por tanto que $\mu$ debe de ser irreducible. 
\end{proof}


\textbf{Contraejemplo de que el recíproco no es cierto.}

Tomamos el espacio vectorial $\R^2$, sobre el cuerpo $\R$ y la aplicación lineal identidad $1: \R^2 \longrightarrow \R^2$. 

Está claro que todos los subespacios vectoriales son $1-$invariantes. 

Además se tiene que el polinomio mínimo de $1$ es $\mu = x - 1$, ya que para cualquier $v \in \R^2$ 
$(x-1) \cdot v = x \cdot x - 1 \cdot v = T(v) - v = 0$ y es irreducible.


\end{document}