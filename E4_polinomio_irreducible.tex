\documentclass{article}

% Esto es para poder escribir acentos directamente:
\usepackage[utf8]{inputenc}
\usepackage[spanish]{babel}


% Paquetes de la AMS:
\usepackage{amsmath, amsthm, amsfonts, amssymb}

% Teoremas
%--------------------------------------------------------------------------
\newtheorem*{teorema*}{Teorema}
\newtheorem{teorema}{Teorema}

% Atajos.
%------------------------------------------------------------------------
\newcommand{\R}{\mathbb{R}}
\newcommand{\N}{\mathbb{N}}
%-------------------------------------------------------------------------
\title{Ejercicio 4 álgebra moderna}
\author{Blanca Cano Camarero}

\begin{document}
\maketitle

\begin{teorema*}
    Sea $T:V \longrightarrow V$ una transformación lineal en un $K-$espacio vectorial de dimensión finita $V$.
    Se tiene que si $V \neq \{0\}$ y $\{0\}$ y $V$ son los únicos 
    subespacios $T$-invariantes de $V$, entonces el polinomio mínimo de $T$ es irreducible en $K[X]$. El recíproco no es cierto. 
\end{teorema*}   

\begin{proof}
   Sea $\mu$ el polinomio mínimo de $T$, que admitirá una descomposición en factores irreducibles de la forma
   $\mu = \prod^n_i=1 p_i$ con $n \in \N$. 
   Para cada $i \in \{1, \ldots, n\}$ llamamos  $q_i = \frac{\mu}{p_i}$ y definimos
   \begin{equation*}
       V_i = \{ q_i m : m \in K\}
   \end{equation*}
   Cada $V_i$ es un  $k-$submódulo. 

   Pero como los únicos conjuntos son los $T$-invariantes entonces $\mu$ debe de ser irreducible. 
\end{proof}


Contraejemplo de que el recíproco no es cierto. 




\end{document}