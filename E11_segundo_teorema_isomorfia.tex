%%%%%%%%%%%%%%%%%%%%%%%%%%%%%%%%%%%%%%%%%%%%%%%%%%%%%%%%%%%%%%%%%%%%%%%%%%%
%
% Plantilla para un art�culo en LaTeX en espa�ol.
%
%%%%%%%%%%%%%%%%%%%%%%%%%%%%%%%%%%%%%%%%%%%%%%%%%%%%%%%%%%%%%%%%%%%%%%%%%%%

\documentclass{article}

% Esto es para poder escribir acentos directamente:
\usepackage[utf8]{inputenc}
\usepackage[spanish]{babel}


% Paquetes de la AMS:
\usepackage{amsmath, amsthm, amsfonts, amssymb}

% Teoremas
%--------------------------------------------------------------------------
\newtheorem*{teorema*}{Teorema}
\newtheorem{teorema}{Teorema}

% Atajos.
%------------------------------------------------------------------------
\newcommand{\R}{\mathbb{R}}
\newcommand{\N}{\mathbb{N}}
%-------------------------------------------------------------------------
\title{Ejercicio 11}
\author{Blanca Cano Camarero}

\begin{document}
\maketitle

\begin{teorema*}
    Sean $L$, $N$ submódulos de un módulo $M$. Demostrad, a partir del primer Teorema del isomorfismo, que existe un isomorfismo de módulos $\frac{N}{L \cap N}.$
\end{teorema*}   

\begin{proof}
   En efecto, vamos a probar que 
   \begin{equation}\label{1}
       \frac{N}{N \cap L} \simeq \frac{N + L}{L}.
   \end{equation}

   Consideramos la proyección $\pi : M \longrightarrow \frac{M}{L}$ dada por $\pi(m)= m+L.$
   Para $\pi_{|N}$ la restricción en $N$, se tiene que 
   $ker(\pi_{|N}) = N \cap L$, $Img(\pi_{|N})= N+L$ y además por 
   el primer teorema de isomorfía, existe un isomorfismo, llamémosle $\phi_1$ por el cual 
   \begin{equation}\label{2}
       \frac{N}{N \cap L} 
       \simeq
       N+L.
   \end{equation}

   Por otro lado consideramos ahora  $\pi_{|N+L}$la restricción en $N+L$, que satisface que 
   $ker(\pi_{|N+L}) = L$ y 
   $Img(\pi_{|N+L})=\pi(N+L) = (N+L)+L = N+L$. Por el primer teorema de isomorfía existe un isomorfismo $\phi_2$ para el cual 
   \begin{equation}\label{3}
       \frac{N+L}{L} \simeq N+L.
   \end{equation}
    Por la transitividad de la isomorfía y las relaciones (\ref{2}) y (\ref{3}) acabamos de probar la relación (\ref{1}) vía el isomorfismo $\phi = \phi_2^{-1} \circ \phi_1$.


\end{proof}





\end{document}