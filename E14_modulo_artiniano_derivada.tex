\documentclass{article}

% Esto es para poder escribir acentos directamente:
\usepackage[utf8]{inputenc}
\usepackage[spanish]{babel}

%Paquetes de la AMS:
\usepackage{amsmath, amsthm, amsfonts, amssymb}
\usepackage{xcolor} % Para poder escribir con colores 

% Teoremas
%--------------------------------------------------------------------------
\newtheorem*{teorema*}{Teorema}
\newtheorem{teorema}{Teorema}

% Atajos.
%------------------------------------------------------------------------
\newcommand{\R}{\mathbb{R}}
\newcommand{\N}{\mathbb{N}}
\newcommand{\K}{K[X]}
%-------------------------------------------------------------------------
\title{Ejercicio 14}
\author{Blanca Cano Camarero}

\begin{document}
\maketitle

\begin{teorema*}
    Sea $K$ un cuerpo de característica $0$ y $K[x]$ el anillo de polinomios. Consideramos la aplicación lineal $T: K[X] \longrightarrow K[X]$ 
    dada por $T(f) = f'$, 
    la derivada formal de $f'$.
    Esto define sobre $K[X]$ una estructura de $K[X]-$módulo a través de $T$. 

    Queremos pues probar que $K[X]$ con esta estructura es un $K[X]-$ módulo artiniano y que además no es isomorfo al $K[X]-$módulo regular.
\end{teorema*}   

\begin{proof}
    Comenzaremos probando que la estructura de $K[X]-$módulo a través de $T$ es artiniano, viendo que todo 
    $\Gamma \subseteq \mathcal{L}(K[X])$ tiene un elemento minimal.

    \textcolor{red}{Habría que intentar ver que están encajados}
    
    Una vez visto que es artiniano, vamos a comprobar que no es finitamente generado. 
    
    Visto $K[X]$ como un grupo aditivo y puesto que el grado de sus polinomios no está acotado (por tener característica cero,
    $K[X]$ no es finitamente generado
    ya que si lo fuera, consideramos un polinomios de grado mayor que cualquiera de los elementos de la base y este no podría ser representado por elementos del generador.
 
    Como no es finitamente generado no es noetheriano.


    Por otra parte  el submódulo regular de ${_{\K}}\K$ por tener estructura de anillo unitario tiene que sus únicos submódulo son $\{0\}$ y $\K$ 
    ya que para cualquier submódulo $N$ diferente de $\{0\}$ 
    existe $0 \neq m \in N$ al ser $\K$ un anillo de división unitario $m^{-1} \in \K$
    y por ser submódulo debe de cumplir que $m^{-1} m \in N$
    es decir $1 \in N$

    de donde deducimos que $1 \K \subseteq N$ y por tanto $N=\K$. 
    Puesto que solo hay dos submódulo posibles está claro que cualquier subconjunto de elemento de $\mathcal{L}(\K)$ tendrá
    un elemento maximal y otro minimal, es decir en noetheriano y artiniano \textcolor{red}{(en los apuntes tengo que no es artiniano, luego debo de estar razonando mal...)} 
    luego es finitamente generado. 

    Razonando por contradicción veremos que el $K[X]-$módulo a través de $T$ y el módulo regular no pueden ser isomorfos. 

    Supongamos que existe tal isomorfismo $\phi$, 
    tomamos cualquier $x$ perteneciente al $K[X]-$módulo a través de $T$
    y vía el isomorfismo $\phi(x)= y \in {_{\K}}\K$
    por ser ${_{\K}}\K$ finitamente generado para cualquier 
    $y \in {_{\K}}\K$ este se escribirá de manera única a partir 
    de sus generadores $\{g_1, \ldots,g_n\}$, basta considerar
    $\{\phi^{-1}(g_1), \ldots, \phi^{-1}(g_n)\}$ pa ver que 
    $x$ admite una representación a partir de tales generados, luego el $K[X]-$módulo a través de $T$ sería finitamente generado, alcanzando con esto una contradicción.

\end{proof}
\end{document}