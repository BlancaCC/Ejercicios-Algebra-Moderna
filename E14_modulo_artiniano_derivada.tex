%%%%%%%%%%%%%%%%%%%%%%%%%%%%%%%%%%%%%%%%%%%%%%%%%%%%%%%%%%%%%%%%%%%%%%%%%%%
%
% Plantilla para un art�culo en LaTeX en espa�ol.
%
%%%%%%%%%%%%%%%%%%%%%%%%%%%%%%%%%%%%%%%%%%%%%%%%%%%%%%%%%%%%%%%%%%%%%%%%%%%

\documentclass{article}

% Esto es para poder escribir acentos directamente:
\usepackage[utf8]{inputenc}
\usepackage[spanish]{babel}


% Paquetes de la AMS:
\usepackage{amsmath, amsthm, amsfonts, amssymb}

% Teoremas
%--------------------------------------------------------------------------
\newtheorem*{teorema*}{Teorema}
\newtheorem{teorema}{Teorema}

% Atajos.
%------------------------------------------------------------------------
\newcommand{\R}{\mathbb{R}}
\newcommand{\N}{\mathbb{N}}
%-------------------------------------------------------------------------
\title{Ejercicio 14}
\author{Blanca Cano Camarero}

\begin{document}
\maketitle

\begin{teorema*}
    Sea $K$ un cuerpo de característica $0$ y $K[x]$ el anillo de polinomios. Consideramos la aplicación lineal $T: K[X] \longrightarrow K[X]$ 
    dada por $T(f) = f'$, 
    la derivada formal de $f'$.
    Esto define sobre $K[X]$ una estructura de $K[X]-$módulo a través de $T$. 

    Queremos pues probar que $K[X]$ con esta estructura es un $K[X]-$ módulo artiniano y que además no es isomorfo al $K[X]-$módulo regular.
\end{teorema*}   

\begin{proof}
    hola esto es una demostración.
\end{proof}





\end{document}