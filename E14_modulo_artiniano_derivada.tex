\documentclass{article}

% Esto es para poder escribir acentos directamente:
\usepackage[utf8]{inputenc}
\usepackage[spanish]{babel}

%Paquetes de la AMS:
\usepackage{amsmath, amsthm, amsfonts, amssymb}
\usepackage{xcolor} % Para poder escribir con colores 

% Teoremas
%--------------------------------------------------------------------------
\newtheorem*{teorema*}{Teorema}
\newtheorem{teorema}{Teorema}

% Atajos.
%------------------------------------------------------------------------
\newcommand{\R}{\mathbb{R}}
\newcommand{\N}{\mathbb{N}}
\newcommand{\K}{K[X]}
%-------------------------------------------------------------------------
\title{Ejercicio 14}
\author{Blanca Cano Camarero}

\begin{document}
\maketitle

\begin{teorema*}
    Sea $K$ un cuerpo de característica $0$ y $K[x]$ el anillo de polinomios. Consideramos la aplicación lineal $T: K[X] \longrightarrow K[X]$ 
    dada por $T(f) = f'$, 
    la derivada formal de $f'$.
    Esto define sobre $K[X]$ una estructura de $K[X]-$módulo a través de $T$. 

    Queremos pues probar que $K[X]$ con esta estructura es un $K[X]-$ módulo artiniano y que además no es isomorfo al $K[X]-$módulo regular.
\end{teorema*}   

\begin{proof}

    La estructura de $K$-módulo viene dada de la siguiente manera 
    \begin{equation}\label{eq:k-modulo}
        p \cdot q = \sum^{grado(p)}_{g = 0} p_i T^{(g)}(q)
    \end{equation}
    Donde $T^{(g)}$ representa la derivada $g$-ésima y $p_i \in K$ son los respectivos coeficientes del polinomio $p = \sum^{grado(p)}_{g = 0} p_i x^g$. 

    Comenzaremos probando que la estructura de $K[X]-$módulo a través de $T$ es artiniano, viendo que todo 
    $\Gamma \subseteq \mathcal{L}(K[X])$ tiene un elemento minimal.
    Observemos que por (\ref{eq:k-modulo}) y ser $K$ un cuerpo, necesariamente todo submódulo será de la forma 
    \begin{equation}
        N_g = 
        \{
            p \in K[x] \quad |  \quad grado(p) < g
        \}
    \end{equation}
    Donde $N_0 = \{0\}$. 

    Así pues, para cualquier conjunto $\Gamma$ tendrá un elemento minimal que será $N_m$ con $m = \min\{ g \in \N : N_g \in \Gamma\}.$


    Por otra parte se tiene que 
    $K[X]-$módulo regular no es artiniano, ya que los subgrupos de la forma 
    $<x^n>$ son submódulo y definen la cadena de composición 
    \begin{equation} \label{eq:no-estabiliza}
        \ldots \subset <x^{n+1}>
        \subset <x^{n}>
        \ldots 
        \subset <x^{2}>
        \subset <x>
        \subset K[x]
    \end{equation}
    que no se estabiliza. 

    Veamos ahora que no existe un isomorfismo $\phi$ entre  $K[X]-$módulo regular 
    y $K[x]$ con la con la estructura definida a través de $T$, ya que de ser así, puesto que si $N$ es un submódulo de 
    $K[X]-$módulo regular $\phi(N)$ lo sería de  $K[x]$ con la con la estructura definida a través de $T$ 
    entonces tomando isomorfimo en cada factor de la cadena (\ref{eq:no-estabiliza}) obtendríamos una cadena que no se estabiliza en $K[x]$ con la con la estructura definida a través de $T$ lo cual contradice que sea artiniano, llegando con esto a una contradicción.


\end{proof}
\end{document}